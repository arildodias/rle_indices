\documentclass[]{article}
\usepackage{lmodern}
\usepackage{amssymb,amsmath}
\usepackage{ifxetex,ifluatex}
\usepackage{fixltx2e} % provides \textsubscript
\ifnum 0\ifxetex 1\fi\ifluatex 1\fi=0 % if pdftex
  \usepackage[T1]{fontenc}
  \usepackage[utf8]{inputenc}
\else % if luatex or xelatex
  \ifxetex
    \usepackage{mathspec}
  \else
    \usepackage{fontspec}
  \fi
  \defaultfontfeatures{Ligatures=TeX,Scale=MatchLowercase}
\fi
% use upquote if available, for straight quotes in verbatim environments
\IfFileExists{upquote.sty}{\usepackage{upquote}}{}
% use microtype if available
\IfFileExists{microtype.sty}{%
\usepackage{microtype}
\UseMicrotypeSet[protrusion]{basicmath} % disable protrusion for tt fonts
}{}
\usepackage[margin=1in]{geometry}
\usepackage{hyperref}
\hypersetup{unicode=true,
            pdftitle={Ecosystem Area Index},
            pdfauthor={Jessica A. Rowland},
            pdfborder={0 0 0},
            breaklinks=true}
\urlstyle{same}  % don't use monospace font for urls
\usepackage{color}
\usepackage{fancyvrb}
\newcommand{\VerbBar}{|}
\newcommand{\VERB}{\Verb[commandchars=\\\{\}]}
\DefineVerbatimEnvironment{Highlighting}{Verbatim}{commandchars=\\\{\}}
% Add ',fontsize=\small' for more characters per line
\usepackage{framed}
\definecolor{shadecolor}{RGB}{248,248,248}
\newenvironment{Shaded}{\begin{snugshade}}{\end{snugshade}}
\newcommand{\KeywordTok}[1]{\textcolor[rgb]{0.13,0.29,0.53}{\textbf{#1}}}
\newcommand{\DataTypeTok}[1]{\textcolor[rgb]{0.13,0.29,0.53}{#1}}
\newcommand{\DecValTok}[1]{\textcolor[rgb]{0.00,0.00,0.81}{#1}}
\newcommand{\BaseNTok}[1]{\textcolor[rgb]{0.00,0.00,0.81}{#1}}
\newcommand{\FloatTok}[1]{\textcolor[rgb]{0.00,0.00,0.81}{#1}}
\newcommand{\ConstantTok}[1]{\textcolor[rgb]{0.00,0.00,0.00}{#1}}
\newcommand{\CharTok}[1]{\textcolor[rgb]{0.31,0.60,0.02}{#1}}
\newcommand{\SpecialCharTok}[1]{\textcolor[rgb]{0.00,0.00,0.00}{#1}}
\newcommand{\StringTok}[1]{\textcolor[rgb]{0.31,0.60,0.02}{#1}}
\newcommand{\VerbatimStringTok}[1]{\textcolor[rgb]{0.31,0.60,0.02}{#1}}
\newcommand{\SpecialStringTok}[1]{\textcolor[rgb]{0.31,0.60,0.02}{#1}}
\newcommand{\ImportTok}[1]{#1}
\newcommand{\CommentTok}[1]{\textcolor[rgb]{0.56,0.35,0.01}{\textit{#1}}}
\newcommand{\DocumentationTok}[1]{\textcolor[rgb]{0.56,0.35,0.01}{\textbf{\textit{#1}}}}
\newcommand{\AnnotationTok}[1]{\textcolor[rgb]{0.56,0.35,0.01}{\textbf{\textit{#1}}}}
\newcommand{\CommentVarTok}[1]{\textcolor[rgb]{0.56,0.35,0.01}{\textbf{\textit{#1}}}}
\newcommand{\OtherTok}[1]{\textcolor[rgb]{0.56,0.35,0.01}{#1}}
\newcommand{\FunctionTok}[1]{\textcolor[rgb]{0.00,0.00,0.00}{#1}}
\newcommand{\VariableTok}[1]{\textcolor[rgb]{0.00,0.00,0.00}{#1}}
\newcommand{\ControlFlowTok}[1]{\textcolor[rgb]{0.13,0.29,0.53}{\textbf{#1}}}
\newcommand{\OperatorTok}[1]{\textcolor[rgb]{0.81,0.36,0.00}{\textbf{#1}}}
\newcommand{\BuiltInTok}[1]{#1}
\newcommand{\ExtensionTok}[1]{#1}
\newcommand{\PreprocessorTok}[1]{\textcolor[rgb]{0.56,0.35,0.01}{\textit{#1}}}
\newcommand{\AttributeTok}[1]{\textcolor[rgb]{0.77,0.63,0.00}{#1}}
\newcommand{\RegionMarkerTok}[1]{#1}
\newcommand{\InformationTok}[1]{\textcolor[rgb]{0.56,0.35,0.01}{\textbf{\textit{#1}}}}
\newcommand{\WarningTok}[1]{\textcolor[rgb]{0.56,0.35,0.01}{\textbf{\textit{#1}}}}
\newcommand{\AlertTok}[1]{\textcolor[rgb]{0.94,0.16,0.16}{#1}}
\newcommand{\ErrorTok}[1]{\textcolor[rgb]{0.64,0.00,0.00}{\textbf{#1}}}
\newcommand{\NormalTok}[1]{#1}
\usepackage{graphicx,grffile}
\makeatletter
\def\maxwidth{\ifdim\Gin@nat@width>\linewidth\linewidth\else\Gin@nat@width\fi}
\def\maxheight{\ifdim\Gin@nat@height>\textheight\textheight\else\Gin@nat@height\fi}
\makeatother
% Scale images if necessary, so that they will not overflow the page
% margins by default, and it is still possible to overwrite the defaults
% using explicit options in \includegraphics[width, height, ...]{}
\setkeys{Gin}{width=\maxwidth,height=\maxheight,keepaspectratio}
\IfFileExists{parskip.sty}{%
\usepackage{parskip}
}{% else
\setlength{\parindent}{0pt}
\setlength{\parskip}{6pt plus 2pt minus 1pt}
}
\setlength{\emergencystretch}{3em}  % prevent overfull lines
\providecommand{\tightlist}{%
  \setlength{\itemsep}{0pt}\setlength{\parskip}{0pt}}
\setcounter{secnumdepth}{0}
% Redefines (sub)paragraphs to behave more like sections
\ifx\paragraph\undefined\else
\let\oldparagraph\paragraph
\renewcommand{\paragraph}[1]{\oldparagraph{#1}\mbox{}}
\fi
\ifx\subparagraph\undefined\else
\let\oldsubparagraph\subparagraph
\renewcommand{\subparagraph}[1]{\oldsubparagraph{#1}\mbox{}}
\fi

%%% Use protect on footnotes to avoid problems with footnotes in titles
\let\rmarkdownfootnote\footnote%
\def\footnote{\protect\rmarkdownfootnote}

%%% Change title format to be more compact
\usepackage{titling}

% Create subtitle command for use in maketitle
\providecommand{\subtitle}[1]{
  \posttitle{
    \begin{center}\large#1\end{center}
    }
}

\setlength{\droptitle}{-2em}

  \title{Ecosystem Area Index}
    \pretitle{\vspace{\droptitle}\centering\huge}
  \posttitle{\par}
    \author{Jessica A. Rowland}
    \preauthor{\centering\large\emph}
  \postauthor{\par}
    \date{}
    \predate{}\postdate{}
  

\begin{document}
\maketitle

\subsection{Index overview}\label{index-overview}

The Ecosystem Area Index (EAI) measures trends measures trends in
changes in ecosystem area towards ecosystem collapse. The EAI is the
geometric mean of the proportion of ecosystem area remaining over a
given timeframe relative to the initial area and an ecosystem-specific
collapse threshold. It uses data on ecosystem area and aea-based
collapse threshold as defined based on IUCN Red List of Ecosystems risk
assessments.

This information sheet provides the code used to calculate the the index
and an example of each step.

\subsection{Set up functions}\label{set-up-functions}

\subsubsection{Calculate the index}\label{calculate-the-index}

The function \emph{calcEAI} selects the column in a dataframe listing
the proportion of the ecosystem area lost over a given timeframe towards
or away from the point where the ecosystem collapses, and percentiles
capturing the middle 95\% of the data.

Parameters are:\\
- eco\_data = dataframe\\
- RLE\_criteria = name of the column with the Red List of Ecosystems
criterion of interest\\
- pct\_change = proportion of the ecosystem area lost over a given
timeframe towards ecosytsem collapse - group1 = the factor (optional)
you want to group the index by. Where not specified, an EAI will be
calculated based on all ecosystems (output = single score)\\
- group2 = the second factor (optional) you want to group the index by

\begin{Shaded}
\begin{Highlighting}[]
\NormalTok{calcEAI <-}\StringTok{ }\ControlFlowTok{function}\NormalTok{(eco_data, RLE_criteria, pct_change, group1, group2)\{}
  
\NormalTok{  filter_data <-}\StringTok{ }\KeywordTok{filter}\NormalTok{(eco_data, RLE_criteria }\OperatorTok{!=}\StringTok{ "NE"} \OperatorTok{&}\StringTok{ }\NormalTok{RLE_criteria }\OperatorTok{!=}\StringTok{ "DD"}\NormalTok{)}
  
  \CommentTok{# Calculate percentage remaining}
\NormalTok{  area_Pct <-}\StringTok{ }\KeywordTok{mutate}\NormalTok{(filter_data, }\DataTypeTok{est_remain =}\NormalTok{ (}\DecValTok{1} \OperatorTok{-}\StringTok{ }\NormalTok{filter_data[[pct_change]]))}
  
  \CommentTok{# Calculate overall index score if missing group, or scores based on a classification if specified}
  \ControlFlowTok{if}\NormalTok{ (}\KeywordTok{missing}\NormalTok{(group1)) \{}
\NormalTok{    values <-}\StringTok{ }\KeywordTok{group_by}\NormalTok{(area_Pct)}
    
\NormalTok{  \} }\ControlFlowTok{else}\NormalTok{ \{}
    \ControlFlowTok{if}\NormalTok{ (}\KeywordTok{missing}\NormalTok{(group2)) \{}
\NormalTok{      values <-}\StringTok{ }\KeywordTok{group_by}\NormalTok{(area_Pct, }\DataTypeTok{group1 =}\NormalTok{ area_Pct[[group1]])}
\NormalTok{    \} }\ControlFlowTok{else}\NormalTok{ \{}
\NormalTok{      values <-}\StringTok{ }\KeywordTok{group_by}\NormalTok{(area_Pct, }\DataTypeTok{group1 =}\NormalTok{ area_Pct[[group1]],}
                                  \DataTypeTok{group2 =}\NormalTok{ area_Pct[[group2]])}
\NormalTok{    \}}
\NormalTok{  \}}

  \CommentTok{# Calculate EAI scores (accounting for zeros) & quantiles}
\NormalTok{  index_scores <-}\StringTok{  }\KeywordTok{summarise}\NormalTok{(values, }\DataTypeTok{total_count =} \KeywordTok{n}\NormalTok{(),}
                             \DataTypeTok{count_non_zeros =} \KeywordTok{length}\NormalTok{(est_remain }\OperatorTok{>}\StringTok{ }\DecValTok{0}\NormalTok{), }\CommentTok{# sample size excluding zeros}
                             \DataTypeTok{EAI =}\NormalTok{ ((}\KeywordTok{exp}\NormalTok{(}\KeywordTok{mean}\NormalTok{(}\KeywordTok{log}\NormalTok{(est_remain[est_remain }\OperatorTok{>}\StringTok{ }\DecValTok{0}\NormalTok{])))) }\OperatorTok{*}\StringTok{ }\NormalTok{(count_non_zeros}\OperatorTok{/}\NormalTok{total_count)), }\CommentTok{# natural log, accounting for zeros}
                             \DataTypeTok{lower =} \KeywordTok{quantile}\NormalTok{(est_remain, }\DataTypeTok{probs =} \FloatTok{0.025}\NormalTok{), }
                             \DataTypeTok{upper =} \KeywordTok{quantile}\NormalTok{(est_remain, }\DataTypeTok{probs =} \FloatTok{0.975}\NormalTok{))}

  \KeywordTok{return}\NormalTok{(index_scores)}
\NormalTok{\}}
\end{Highlighting}
\end{Shaded}

\paragraph{Example}\label{example}

This test data set are available from github.

\begin{Shaded}
\begin{Highlighting}[]
\CommentTok{# Load packages}
\KeywordTok{library}\NormalTok{(dplyr)}

\CommentTok{# Load data}
\NormalTok{data <-}\StringTok{ }\KeywordTok{read.csv}\NormalTok{(}\StringTok{"~/Documents/*Career/*PhD/*Projects/2 - RLE indicators/Github script/Github_example_data_EAI.csv"}\NormalTok{)}

\CommentTok{# View data}
\KeywordTok{head}\NormalTok{(data)}
\end{Highlighting}
\end{Shaded}

\begin{verbatim}
##   ecosystem realm region criterion_A1 area_lost
## 1         1     A      1           A1      0.40
## 2         2     A      1           A1      0.10
## 3         3     B      1           A1      0.05
## 4         4     B      2           A1      0.10
## 5         5     B      2           A1      0.60
## 6         6     A      2           A1      0.12
\end{verbatim}

Calculate the index using no groupings:

\begin{Shaded}
\begin{Highlighting}[]
\CommentTok{# Calculate the index values}
\NormalTok{output <-}\StringTok{ }\KeywordTok{calcEAI}\NormalTok{(data,}
                  \DataTypeTok{pct_change =} \StringTok{"area_lost"}\NormalTok{,}
                  \DataTypeTok{RLE_criteria =} \StringTok{"criterion"}\NormalTok{)}

\CommentTok{# View output}
\KeywordTok{head}\NormalTok{(output)}
\end{Highlighting}
\end{Shaded}

\begin{verbatim}
## # A tibble: 1 x 5
##   total_count count_non_zeros   EAI lower upper
##         <int>           <int> <dbl> <dbl> <dbl>
## 1          13              13 0.857  0.46     1
\end{verbatim}

Calculate the index using one grouping:

\begin{Shaded}
\begin{Highlighting}[]
\CommentTok{# Calculate the index values}
\NormalTok{output_one_grouping <-}\StringTok{ }\KeywordTok{calcEAI}\NormalTok{(data,}
                               \DataTypeTok{pct_change =} \StringTok{"area_lost"}\NormalTok{,}
                               \DataTypeTok{RLE_criteria =} \StringTok{"criterion"}\NormalTok{,}
                               \DataTypeTok{group1 =} \StringTok{"realm"}\NormalTok{)}

\CommentTok{# View output}
\KeywordTok{head}\NormalTok{(output_one_grouping)}
\end{Highlighting}
\end{Shaded}

\begin{verbatim}
## # A tibble: 2 x 6
##   group1 total_count count_non_zeros   EAI lower upper
##   <fct>        <int>           <int> <dbl> <dbl> <dbl>
## 1 A                7               7 0.879 0.638 1    
## 2 B                6               6 0.832 0.462 0.999
\end{verbatim}

Calculate the index using two groupings where ecosystems are grouped by
realm and continent:

\begin{Shaded}
\begin{Highlighting}[]
\CommentTok{# Calculate the index values}
\NormalTok{output_two_groupings <-}\StringTok{ }\KeywordTok{calcEAI}\NormalTok{(data,}
                                \DataTypeTok{pct_change =} \StringTok{"area_lost"}\NormalTok{,}
                                \DataTypeTok{RLE_criteria =} \StringTok{"criterion"}\NormalTok{,}
                                \DataTypeTok{group1 =} \StringTok{"realm"}\NormalTok{,}
                                \DataTypeTok{group2 =} \StringTok{"region"}\NormalTok{)}

\CommentTok{# View output}
\KeywordTok{head}\NormalTok{(output_two_groupings)}
\end{Highlighting}
\end{Shaded}

\begin{verbatim}
## # A tibble: 6 x 7
## # Groups:   group1 [2]
##   group1 group2 total_count count_non_zeros   EAI lower upper
##   <fct>   <int>       <int>           <int> <dbl> <dbl> <dbl>
## 1 A           1           2               2 0.735 0.608 0.892
## 2 A           2           2               2 0.865 0.851 0.879
## 3 A           3           3               3 1     1     1    
## 4 B           1           1               1 0.95  0.95  0.95 
## 5 B           2           2               2 0.6   0.412 0.888
## 6 B           3           3               3 0.990 0.980 1.000
\end{verbatim}


\end{document}
